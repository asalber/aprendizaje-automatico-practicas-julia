\begin{tikzpicture}[
    node distance=1.5cm,
    neuron/.style={circle, draw, minimum size=1cm},
    input/.style={rectangle, draw, minimum size=0.5cm, fill=green!20},
    output/.style={rectangle, draw, minimum size=0.5cm, fill=red!20},
    weight/.style={font=\small, midway, above, sloped}
]

% Input neurons
\node[input] (in1) {1};
\node[input, below of=in1] (in2) {$x_1$};
\node[input, below of=in2] (in3) {$x_2$};

% Output neuron
\node[neuron, right=2cm of in2] (out) {\Large $\sum$};

% Connect inputs to output
\draw[-{Latex[length=2mm]}] (in1) -- (out) node[weight] {$w_0$};
\draw[-{Latex[length=2mm]}] (in2) -- (out) node[weight] {$w_1$};
\draw[-{Latex[length=2mm]}] (in3) -- (out) node[weight] (w2) {$w_2$};

% Activation function
\node[neuron, right=1cm of out] (activation) {$f(x)$};
\draw[-{Latex[length=2mm]}] (out) -- (activation);
\begin{scope}[xshift=4.8cm, yshift=-1.5cm, scale=1]
    \addaxes;
    % flexible selection of activation function
    % \relu
    \stepfunc;
\end{scope};

% Output function 
\node[output, right=1cm of activation] (output) {$y$};
\draw[-{Latex[length=2mm]}] (activation) -- (output);

% Etiquetas
\node[left=5mm of in2, rotate=90, anchor=north] {Entradas};
\node[below=2mm of output] {Salida};
\node[below=2mm of w2] {Pesos};
\node[below=2mm of out, align=center] {Suma\\ponderada};
\node[below=2mm of activation, align=center] {Función de\\activación};
\end{tikzpicture}